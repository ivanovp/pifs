%\documentclass[a4paper,10pt,hungarian]{report}
%\documentclass[a4paper,10pt,hungarian]{article}
\documentclass[a4paper,10pt]{article}
\usepackage{upgreek}
\usepackage{relsize}
%a ,,hungarian'' opci� csak a draftcopy csomaghoz kell!
%\usepackage[latin2]{inputenc}
%\usepackage[T1]{fontenc}
%\usepackage[magyar]{babel}
%\usepackage[left=2.5cm,right=2.5cm,top=2cm,bottom=2cm]{geometry}
\usepackage[left=2.5cm,right=2cm,top=2cm,bottom=2cm]{geometry}
\usepackage{indentfirst}
\sloppy
%\usepackage{graphicx}
%\usepackage{ifthen}
%\usepackage{multicol}
%\usepackage[outline]{draftcopy} %outline,bottom,light,none/first/firsttwo
%\usepacgake{wasysym}
%%%% Don't modify next two line! %%%%
%.PDF.\usepackage{times}
%.PDF.\usepackage[T1]{mathtime}
\frenchspacing
\newcommand{\mypi}{{\larger $\uppi$}}
%\newcommand{\mypi}{{\larger $\pi$}}
\newcommand{\pifs}{\mypi{}fs}
\newcommand{\pifsl}{Pi file system}
\author{Peter Ivanov \textless{}ivanovp@gmail.com\textgreater{}}
\title{\pifsl{} (\pifs{})}
%%
%\usepackage[dvips]{graphicx}

%%

\begin{document}
\maketitle
%\tableofcontents
\section{Introduction}
This file system was developed for embedded systems which use NOR flash as 
storage. 

NOR flash ICs have very low price nowadays (2017) and can be used
to store files. But there are few problems to consider when designing a file
system:
\begin{itemize}
\item NOR flashes can be programmed (set bits to zero) by pages, but only 
can be erased (set bits to one) in a larger quantity, which are mostly called
block in the datasheets. One page is usually 256 or 512 bytes, one block 
consits of 16, 256, 1024, etc. pages. So typical block sizes are 4 KiB, 
64 KiB, 256 KiB.
\item Blocks can be erased ~10,000--100,000 times. After that data retention is 
not guaranteed. Therefore all blocks should be erased uniformly. This method 
is called wear leveling.
\end{itemize}

\pifs{} can be scaled from 4 Mbit (512 KiB) to 256 Mbit (32 MiB) memory 
sizes.

The development started as teach-myself project and I released the code in 
hope that it will be useful for someone else as well.

\section{Features of the \pifs{}}
\begin{itemize}
\item Small memory footprint
\item Size of logical page is user-defined
\item Cache buffer for page (currently only one page is cached)
\item Directory handling
\item Dynamic wear-leveling
\item Static wear-leveling (limited, work-in-progress)
\item User data can be added for files: permissions, owner IDs, etc.
\item At the beginning of flash memory reserved blocks can be defined, which 
are not used by the file system
\end{itemize}

\section{Limitations of the \pifs{}}
\begin{itemize}
\item Only one NOR flash chip can be used
\item Memory and file system configuration cannot be changed during run-time
\item One directory can only store pre-defined number of files or directories
\end{itemize}

%\subsection{Alc�m}
\end{document}
